\begin{resumo}[Abstract]

 \begin{otherlanguage*}{english}
  Parkinson's disease is a neurological disease that affects a large portion of the world's population, most of whom are elderly. Although the actual reasons that cause the disease have not been discovered, its diagnosis is possible thanks to the evident signs caused by the presence of the disease, being sufficient. There is still a need to obtain more information about the state of the patient of this disease for possible studies about it. In order to guarantee access to this information and to assist in the diagnosis of Parkinson's Disease is possible to use the concepts of Machine Learning, which has been widely requested in several areas of knowledge, including health, and to solve problems of several contexts. This course completion work implements a system called DPDP (Preliminary Detector of Parkinson's Disease) in which a Machine Learning model is applied to aid in the preliminary diagnosis of Parkinson's disease. This was done using samples collected by (sEMG), in order to recognize patterns and to obtain new information about the disease in question. The system uses the algorithm Random Forest to detect the patterns of the disease, since was more successful in comparison with other models studied in this work.

   \vspace{\onelineskip}
 
   \noindent 
   \textbf{Key-words}: Parkinson; Machine Learning; sEMG.
 \end{otherlanguage*}
\end{resumo}
