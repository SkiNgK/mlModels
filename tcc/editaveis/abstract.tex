\begin{resumo}[Abstract]

  A Doença de Parkinson é uma doença neurológica que atinge uma grande parcela da população mundial sendo majoritariamente pessoas idosas, apesar de não terem sido descoberto os motivos efetivos que causam a doença seu diagnóstico é possível graças aos sinais evidentes causados pela presença da mesma, apesar dos meios para validar este diagnóstico já serem dados como suficientes ainda existe uma necessidade de se obter mais informações sobre o estado do paciente portador dessa doença para possíveis estudos sobre a mesma. Para garantir o acesso à essas informações e auxiliar no diagnóstico da Doença de Parkinson é possível utilizar os conceitos de Aprendizado de Máquina que tem sido muito requisitado em diversas áreas do conhecimento, incluindo a área da saúde, para a solução de problemas de diversos contextos. Este trabalho de conclusão de curso visa implementar um modelo de Aprendizado de Máquina que auxilie no diagnóstico da Doença de Parkinson utilizando amostras recolhidas por dispositivos sEMGs afim de reconhecer padrões e obter novas informações sobre a doença em questão, utilizando uma proposta de metodologia de construção de sistema de Aprendizado de Máquina.

 \begin{otherlanguage*}{english}
  Parkinson's disease is a neurological disease that affects a large part of the world population, being mostly elderly people, although the effective reasons that cause the disease have not been discovered, its diagnosis is possible thanks to the evident signs caused by the presence of the disease, despite the means to validate this diagnosis are already given as sufficient there is still a need to obtain more information about the state of the patient with this disease for possible studies on it. To guarantee access to this information and to assist in the diagnosis of Parkinson's Disease, it is possible to use the concepts of Machine Learning that have been widely requested in several areas of knowledge, including health, for the solution of problems in different contexts. This course completion work aims to implement a Machine Learning model that assists in the diagnosis of Parkinson's Disease using samples collected by sEMGs devices in order to recognize patterns and obtain new information about the disease in question, using a proposed methodology of construction of machine learning system.

   \vspace{\onelineskip}
 
   \noindent 
   \textbf{Key-words}: latex. abntex. text editoration.
 \end{otherlanguage*}
\end{resumo}
