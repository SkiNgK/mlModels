\chapter{Parkinson}
\section{Definição}
A Doença de Parkinson (DP) é caracterizada por ser uma doença crônica, progressiva e degenerativa do sistema nevorso central, reconhecida principalmente por distubios motores \cite{souzametodos}, como a bradicinesia, tremores corpoais em repouso e rigidez corporal \cite{da2016aspectos}, A bradicinesia é a redução da movimentação ou seja o indivíduo consegue se movimentar porém lentamente. A DP é resultante da degeneração dos neurônios dopaminérgicos, responsáveis pela produção de dopamina um neurotransmissor fundamental para a gestão dos movimentos. 

\section{Diagnóstico e Sintomas}
A doença de Parkinson afeta aproximadamente entre 1\% a 2\% da população mundial acima dos 65 anos, sendo que no Brasil atinge entorno de 3\% da população nesta faiza etaria. \cite{magalhaes2009descobrindo}, tendo a idade como o único fator de risco conhecido, a DP é rara antes dos 40 anos, aumenta após os 50 e é tem maior probabilidade após os 70 anos \cite{peixinho2006alteraccoes}, porém a incidência da doença não esta restrita somente a pessoas idosas, uma vez que 20\% dos casos são de pessoas com menos de 50 anos. \cite{gago2014manual}

Com relação a incidência por sexo não há concenco estabelecido, mas alguns estudos relacionam ser um pouco maior a ocorrência no sexo masculino \cite{peixinho2006alteraccoes}.

O diagnóstico da DP atualmente é clinico reaçizado pelo Neurologista, onde identifica-se a bradicinésia (lentidão dos movimentos) e pelo menos um de três outros sintomas, tremor em repouso, rigidez muscular ou instabilidade postural \cite{gago2014manual}. Também pode ser utilizado para confirmar a DP exames como Tomográfia computadoriazada e ressonância magnética cerebral dentre outros a ser prescrito pelo Neurologista \cite{gago2014manual}.

Outros sintomas reconrentes são associadoas a comunicação oral, onde 90\% dos pacientes possuem problemas relacionados a fala, como gagueira, rouquidão, alteração ou enfraquecimento da voz \cite{zarzur2010laryngeal}, outro sintoma relevante é a demencia que cerca de 25\% dos pacientes \cite{pamplona1996demencia}\%, também podém ocorrer alterações do sono de mémoria e depressão\cite{barbosa2005parkinsons}.

\section{DP e sEMG}


\section{DP e machine learning}