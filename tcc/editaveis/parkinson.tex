\chapter{Parkinson}
\section{Definição}
A Doença de Parkinson (DP) é caracterizada por ser uma doença crônica e degenerativa do sistema nevorso central, reconhecida principalmente por distúrbios motores \cite{souzametodos}, como a bradicinesia (redução da movimentação, o indivíduo consegue se movimentar porém lentamente), tremores corpoais em repouso e rigidez corporal \cite{da2016aspectos}. A DP é resultante da degeneração dos neurônios dopaminérgicos, responsáveis pela produção de dopamina um neurotransmissor fundamental para a gestão dos movimentos. 

\section{Diagnóstico e Sintomas}
A doença de Parkinson afeta aproximadamente entre 1\% a 2\% da população mundial acima dos 65 anos, sendo que no Brasil atinge entorno de 3\% da população nesta faiza etaria. \cite{magalhaes2009descobrindo}, tendo a idade como o único fator de risco conhecido, a DP é rara antes dos 40 anos, aumenta após os 50 e é máxima após os 70 anos \cite{peixinho2006alteraccoes}, porém a incidência da doença não esta restrita somente a pessoas idosas, uma vez que 20\% dos casos são de pessoas com menos de 50 anos. \cite{gago2014manual}

Com relação a incidência por sexo não há concenco estabelecido, mas alguns estudos relacionam ser um pouco maior a ocorrência no sexo masculino \cite{peixinho2006alteraccoes}.

O diagnóstico da DP atualmente é clinico reaçizado pelo Neurologista, onde identifica-se a bradicinésia (lentidão dos movimentos) e pelo menos um de três outros sintomas, tremor em repouso, rigidez muscular ou instabilidade postural \cite{gago2014manual}. Também pode ser utilizado para confirmar a DP exames como Tomográfia computadoriazada e ressonância magnética cerebral dentre outros a ser prescrito pelo Neurologista \cite{gago2014manual}.

Outros sintomas reconrentes são associadoas a comunicação oral, onde 90\% dos pacientes possuem problemas relacionados a fala, como gagueira, rouquidão, alteração ou enfraquecimento da voz \cite{zarzur2010laryngeal}, outro sintoma relevante é a demencia que cerca de 25\% dos pacientes \cite{pamplona1996demencia}\%, também podém ocorrer alterações do sono de mémoria e depressão\cite{barbosa2005parkinsons}.

\section{sEMG e machine learning}
A utilização de Aprendisado de máquinas em conjunto com sEMG para o auxilio na detecção de alguma doença, já é um tópico bem explorado possuindo diversas publicações relacionadas a esta área de estudo, na base de dados Capes a string de busca '(emg AND (Machine learning))' retornou 4.217 resultados, no \textit{ScienceDirect} a mesma \textit{string} retornou 2.396 \section{Parkinson e sEMG} e no \textit{ieeexplore} 169.

Como observado na Revista acadêmica de \cite{yousefi2014characterizing}, os transtornos neuromusculares podem são comumente separados em Miopatia e Neuropatia, onde Miopatia refere-se a um grupo de patologias que atigem diretamente o tecido muscular sem relação com disfunções do sistema nevorso. Já a Neuropatia, onde encontra-se a doença de Parkinson, é caracterizada por qualquer dano nos nervos envolvidos no controle muscular.

Em resumo as diversas tecnicas de Aprendizado de máquinas sobre sEMG, pode ser catalogadas em três etapas os dados devem ser analisados, decompostos e classificados\cite{yousefi2014characterizing}.



\section{Parkinson, sEMG e machine learning}