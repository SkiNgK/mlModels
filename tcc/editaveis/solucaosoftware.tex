\chapter[Solução de \textit{software}]{Solução de software}
Esse projeto utilizará a metodologia ágil de desenvolvimento de \textit{software} com uma adaptação do \textit{Scrum} para projetos pequenos, para a organização e desenvolvimento do trabalho como um todo.

\section{Ágil}
\subsection{\textit{Scrum}}
O \textit{Scrum} é uma metodologia ágil para o gerenciamento de projeto de \textit{software}, que consiste na estruturação de um projeto em pequenas fases incrementais, com prazo definido e escopo variável de acordo as demandas e necessidades do cliente. As iterações são chamados de \textit{Sprints} e normalmente vão de uma semana a um mês, dependendo da organização e necessidades do projeto. As funcionalidades do \textit{software} a serem desenvolvidas, são colocadas em uma lista denominada \textit{Product Backlog}. Em cada \textit{Sprint} as funcionalidade são priorizas e selecionas, e após são alocadas na lista de \textit{Sprint Backlog}, as quais devem ser desenvolvidas na \textit{Sprint} \cite{sutherland2016scrum}.

Durante as \textit{Sprints}, devem ocorrer reuniões diárias, as quais necessitam ser objetivas e rápidas, com tempo fechado, não sendo ideal ultrapassar 15 minutos. Nessa reunião, busca-se alinhar o desenvolvimento do projeto entre toda a equipe, onde todos os integrantes devem responder as seguintes questões \cite{sutherland2016scrum}.:
\begin{itemize}
    \item \textit{O que eu fiz ontem?}
    \item \textit{O que eu farei hoje?}
\end{itemize}

Ao final de cada \textit{Sprint}, devem ocorrer uma reunião de retrospectiva, onde a equipe planejará a nova \textit{Sprint} \cite{sutherland2016scrum}.

\section{Ferramentas}
Como descrito na seção ~\ref{sec:MAFerramentas} na página \pageref{sec:MAFerramentas}, uma das principais linguagens de programação em AM é o \textit{Python}, com a utilização da biblioteca \textit{scikit-learn}. Com este cenário em vista, optou-se por desenvolver o resto da aplicação em \textit{Python}. Deste modo, deve-se obter uma boa integração com o modelo desenvolvido.

\subsection{Scikit-learn}
O \textit{scikit-learn} será o \textit{framework} utilizado para o desenvolvimento dos modelos de AM.
\subsection{Django-ResT}

A \textit{Django Rest Framework} é uma biblioteca em \textit{Python} que viabiliza de forma simples a criação de APIs com o padrão arquitetural REST \textit{(sigla do inglês: Representational State Transfer, Transferência de Estado Representacional)} em projetos Django usando padrões estabelecidos para proporcionar produtividade na criação de APIs (sigla do inglês: \textit{Application Programming Interface}, Interface de Programação de Aplicativos). Sendo que, REST é um padrão arquitetural da engenharia de \textit{software} utilizado para a comunicação entre diferentes sistemas, e Django é um \textit{framework} para a criação de páginas \textit{web} \cite{christie2011django}.

\subsection{React}

A \textit{React} é uma biblioteca JavaScript declarativa criada pelo \textit{Facebook}. Sendo bem eficiente e flexível para a criação de interfaces de usuário. Ou seja, ela é simplesmente uma coleção de funcionalidades que podem ser chamadas pelo desenvolvedor, para desenvolver uma interface de usuário reaproveitável \cite{reactjs}.

\section{Produto Mínimo Viável}

Um MVP (sigla do inglês: \textit{minimum viable product}, produto viável mínimo) é uma técnica de desenvolvimento na qual um novo produto ou site é desenvolvido com recursos suficientes para satisfazer os primeiros usuários. O conjunto final e completo de recursos é projetado e desenvolvido somente após considerar o feedback dos usuários iniciais do produto \cite{MVP}.

A solução de \textit{software} presente neste trabalho de conclusão de curso é um MVP que consiste no núcleo principal de um possível produto final a ser desenvolvido. São especificados no Documento de Visão, documento apresentado no Apêndice \ref{adocvisao}, os requisitos que atendem as principais necessidades dos usuários vide ao que foi proposto neste trabalho. Além disso o Apêndice Documento de Visão apresenta os perfis do próprio usuário no qual o MVP do produto será destinado e das partes interessadas no projeto como um todo.
