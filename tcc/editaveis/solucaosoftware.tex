\chapter[Solução de software]{Solução de software}
Esse projeto utilizará a metódologia ágil de desenvolvimento de software com uma adaptação do \textit{Scrum} para projetos pequenos, para a organização e desenvolvimento do trabalho como um todo.

\section{Ágil}
\subsection{Scrum}
O \textit{Scrum} é um metódologia ágil para o gerenciamento de projeto de software, consiste na estruturação de um projeto em pequenas fases icrementais, com prazo definido e esopo variavel de acordo as demandas e necessidades do cliente. As iteraçãoes são chamados de \textit{Sprints} e normalmente vão de uma semana a um mês, dependendo da organização e necessidades do projeto. As funcionalidade do software a serem desenvolvidas, são colocadas em uma lista denominada \textit{Product Backlog}. Em cada \textit{Sprint} as funcionalidade são priorizas e selecionas, e após são alocadas na lista de \textit{Sprint Backlog}, as quais seram desenvolvidas na \textit{Sprint} \cite{sutherland2016scrum}.

Durante as \textit{Sprints}, devem ocorrer reuniões diarias, as quais necessitam ser objetivas e rápidas, com tempo fechado, não sendo idel ultrapassar 15 minutos. Nessa reunião, busca-se alinhar o desenvolvimento do projeto entre toda a equipe, onde todos os integrantes devem responder as seguintes questões \cite{sutherland2016scrum}.:
\begin{itemize}
    \item \textit{O que eu fiz ontem?}
    \item \textit{O que eu farei hoje?}
\end{itemize}

Ao final de cada \textit{Sprint}, devem ocorrer uma reunião de retrospectiva, onde a equipe planejará a nova \textit{Sprint} \cite{sutherland2016scrum}..

\section{Ferramentas}

\subsection{Django-ResT}
\subsection{React}
\subsection{scikit-learn}
\subsection{React}
