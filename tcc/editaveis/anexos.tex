\begin{anexosenv}

	\partanexos
	\includepdf[pages=-,pagecommand=\chapter{Protocolo PKS}]{./anexanexospk/Protocolo_PKS.pdf}

	\chapter{Documento de Arquiteruta}
	\section{introdução}
	Esta documento descreve a arquitetura e tecnologias da aplicação, utilizadas no desenvolvido do sistema de auxilio na detecção da doença de Parkinson utilizando Tremor em repouso. Desenvolvido pelos alunos de Trabalho de Conclusão de Curso 2 da Universidade de Brasília.

	Será desenvolvida uma aplicação web, podendo facilmente ser migrada para uma aplicação mobile em projetos futuros.

	\section{Tecnologias}
	\subsection{Single Page Aplication - SPA}

	\subsection{REST}
	Baseado no baseado no protocolo de comunicação de rede HTTP, sendo simples, rápida e com uma fácil comunicação entre clientes e servidores. Funciona com objetos em formato JSON e os métodos definidos no protocolo HTTP ( POST, GET, PUT e DELETE) possibilitando assim uma troca rápida de informação.

	\section{Representação da Arquitetura}
	\section{Restrições e Metas Arquiteturais}
	\section{Visão Lógica}
	\subsection{Diagrama de Classes}
	\subsection{Diagrama de Pacotes}
	\section{Referências}

\end{anexosenv}

