\chapter[Introdução]{Introdução} 
\section{Preâmbulo} 
Este trabalho visa o desenvolvimento de um software, capaz de auxiliar o diagnóstico da Doença de Parkinson, utilizando aprendizado de máquinas para identificar se o indivíduo é portador ou não da doença. Com relação aos dados que serão analisados pelo algorítimo, estes dados serão coletados através de sEMG. 

A DP (Doença de \textit{Parkinson}) é uma doença crônica que afeta o sistema nervoso central, prejudicando a mobilidade dos doentes. A DP possue três principais sintomas a bradicinesia, tremores corporais em repouso e rigidez corporal, a doença não possui cura e sua causa ainda é desconhecida, deste modo o tratamento é focado na melhora de qualidade de vida dos pacientes\cite{da2016aspectos}. 

A sEMG (sigla do inglês: \textit{surface electromyography}, eletromiografia de superfície) é uma técnica rápida e não invasiva para monitorar a atividade elétrica envolvendo as fibras musculares, através do uso de eletrodos sobre a pele do paciente, gerando um sinal eletromiográfico. Um sinal é o somatório das contrações musculares identificadas pelos eletrodos \cite{de2010eletromiografia}. 

A AM (sigla do inglês: \textit{machine learning}, aprendizado de máquinas) é a uma área da inteligência artificial e da ciência de dados, onde estuda-se o desenvolvimento de algoritmos autônomos capazes de aprender com os seus erros e realizar predições sobre novos dados. Se executado corretamente obtém um alto nível de acerto, muitas vezes superior ao realizado por humanos \cite{Kohavi} 

O SVM (sigla do inglês: \textit{support vector machine}, Máquina de Vetores de Suporte) é um algoritmo de AM, amplamente utilizado devido a eficiência, rapidez e facilidade de implementação. SVM é utilizado em AM, para classificação, regressão de dados e também na detecção de \textit{outliers} (valores atípicos) \cite{scikit-learn}. 

\section{Justificativa} 
Atualmente o diagnóstico da Doença de Parkinson é basicamente clínico, onde um neurologista identifica alguns sintomas associados a doença, como a bradicinesia, tremor em repouso, rigidez muscular e instabilidade postural \cite{gago2014manual}. A sEMG pode auxiliar esse processo, porém a onda gerada não é de fácil interpretação, sendo que com AM é possível obter um resultado rápido e com alta precisão, como demonstrado por \citeonline{camara2015resting}, onde obteve-se algo em torno de 98\% de acerto. 

Logo, com o desenvolvimento do\textit{software} proposto neste trabalho, este processo de diagnóstico poderá ser facilitado, utilizando um único exame não invasivo de sEMG, combinado com AM para classificar os portadores de DP. Auxiliando com rapidez e eficácia a detecção da DP. 

\section{Objetivos} 
\subsection{Objetivo Geral} 
O Objetivo deste Trabalho é o desenvolvimento de um \textit{software} utilizando algorítimos de AM capazes de auxiliar no diagnóstico da DP, classificando os indivíduos em portadores da DP e não portadores. 

\subsection{Objetivos Específicos}
\begin{itemize}
    \item Pesquisar trabalhos semelhantes;
    \item Estudar o sinal sEMG relacionado a DP;
    \item Tratar os dados para otimizar o processo de AM;
    \item Escolher algorítimos para resolver o problema de AM;
    \item Definir \textit{features} para análise no \textit{software};
    \item Treinar o modelo para a predição;
    \item Validar a o modelo;
    \item Definir arquitetura do \textit{software};
    \item Desenvolver um sistema integrando o algorítimo treinado anteriormente, com um \textit{software} amigável ao usuário;
    \item Testar o sistema com portadores da DP;
\end{itemize}