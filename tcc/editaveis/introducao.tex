\chapter[Introdução]{Introdução}
\section{Preâmbulo}
Parkinson é uma doença cronica que afeta o sistema nervoso central prejudicando a mobilidade dos doentes, possue três principais sintomas a bradicinesia, tremores corporais em repouso e rigidez corporal, também não possui cura e sua causa ainda é desconhecida deste modo o tratamento é focado na melhora de qualidade de vida dos pacientes\cite{da2016aspectos}.

A eletromiografia de superfície (sEMG) é uma técnica rápida e não invasiva para monitorar a atividade elétrica envolvendo as fibras musculares, através do uso de eletrodos sobre a pele do paciente, gerando uma sinal eletromiográfico que é um somatório das contrações musculares identificadas pelos eletrodos \cite{de2010eletromiografia}.

Aprendizado de Máquinas (AM) é a uma área da inteligência artificial e da ciência de dados, onde estuda o desenvolvimento de algorítimos autônomos capazes de aprender com os seus erros e realizar predições sobre novos dados, se executado corretamente obtém um alto nível de acerto, muitas vezes superior ao realizado por humanos \cite{Kohavi}

Máquina de Vetores de Suporte (SVM, do inglês: \textit{support vector machine}) é um algorítimo de AM, utilizado para classificar dados, amplamente utilizado devido a eficiência, rapidez e segurança associadas.

\section{Justificativa}
Atualmente o diagnóstico da Doença de Parkinson é basicamente clinico onde o Neurologista identifica alguns sintomas associados a doença \cite{gago2014manual}, a sEMG pode auxiliar esse processo, porém a onda gerada não é de fácil interpretação. Assim sendo, este sistema visa auxiliar o processo de diagnóstico utilizando um único exame não invasivo com sEMG e AM, obtendo um resultado quase imediato.

\section{Objetivos}
\subsection{Objetivo Geral}
O Objetivo deste Trabalho é o desenvolvimento de um \textit{Software} utilizando algorítimos de aprendizado de máquinas capazes de auxiliar no diagnóstico da Doença de Parkinson, com dados obtidos por um dispositivo de coleta de sinais sEMG.

\subsection{Objetivos Específicos}
\begin{itemize}
    \item Analisar pesquisas semelhantes.
    \item Entender o sinal sEMG relacionado a DP.
    \item Definir \textit{features} para análise no \textit{Software}.
    \item Tratar os dados para otimizar o processo de AM.
    \item Escolher algorítimos para resolver o problema de AM.
    \item Treinar os algorítimos para a predição.
    \item Validar a o modelo.
    \item Definir arquitetura do \textit{Software}.
    \item Desenvolver um sistema integrando o algorítimo treinado anteriormente.
    \item Testar o \textit{software} desenvolvido.
\end{itemize}
\section{Trabalhos correlatos}