\chapter{Metodologia}
\section{Processo de construção de uma aplicação de Aprendizado de Máquina}

Segundo o guia do desenvolvedor de aprendizado de máquina da Amazon e o livro “Hands-On Machine Learning with Scikit-Learn & TensorFlow, Concepts, Tools, and Techniques to Build Intelligent Systems” escrito por Aurélien Géron, o processo de desenvolvimento de uma aplicação de AM é interativo e envolve uma série sequencial de passos que de maneira geral vão desde entender a “big picture” até a apresentação da solução de proposta da aplicação de AP. O processo pode ser visualizado pela representação Business Process Model and Notation - BPNM - da figura:

Figura do processo

\subsection{Abstrair o problema no contexto}

Entender o principal problema de AM do contexto e a natureza desse problema observando o quais respostas o modelo escolhido deverá predizer. Formular o problema é primeiro passo para decidir o que é desejado predizer.

Imagine o seguinte cenário, no qual você deseja fabricar produtos, porém sua decisão de qual produto fabricar depende do número de vendas em potencial. Nesse cenário, você precisa saber a quantidade de vezes que cada produto foi comprado, ou seja, predizer o número de vendas. Existem diversas maneiras de definir esse problema utilizando AM, porém a escolha de como definir o problema depende do seu caso de uso ou necessidade comercial \cite{Amazon}.

Em termos mais técnicos dentro do AM, o ponto importante do entendimento do contexto e a abstração do problema, é que essa etapa tem um papel fundamental na escolha dos modelos, pois tendo o entendimento da natureza do problema será selecionado o modelo para predizer os resultados para esse dado problema com base no tipo de aprendizado mais adequado a este contexto.

\subsection{Coleta de dados}

Os dados coletados nessa etapa devem estar ligado diretamente ao que é requerido pelo problema e atender esses requisitos a fim de atender as necessidades do contexto é de suma importância, pois alguns dos já citados desafios encontrados no desenvolvimento de sistema de AM estão relacionados aos dados que serão utilizados \cite{Aurélien}

A quantidade de dados coletados nessa etapa ainda é incerta, porém é levado em consideração que quanto maior a quantidade de dados coletados, melhor. Neste trabalho foram utilizados sinais neuromusculares obtidos por aparelhos denomidados sEMG, assim como foi demonstrado no (...)

\subsection{Processar Dados}

\subsection{Selecionar o Modelo}

\subsection{Treinar o Modelo}

\subsection{Apresentar a solução}

\subsection{Sinais sEMG}
\subsection{Classificações}
\subsection{Pré-processamento}