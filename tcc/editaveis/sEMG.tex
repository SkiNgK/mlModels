\chapter{Eletromiografia de Superfície - sEMG}
\section{Definição}
A eletromiográfia de Superfície é o estudo relacionado as transformações elétricas referentes as contrações musculares. É um exame indolor e não invasivo permitindo assim a execução com mobilidade dos movimentos musculares solicitados, podendo ser execultada repetidas vezes sem causar um grande desconforto ao paciente, sendo rápida, barata, livre de radiação, não invasivo e de fácil compreensão, podendo ser utilizado na análise de um grupo ou um feixe muscular específico \cite{de2010eletromiografia}.

Um sinal sEMG é um sinal obtido na medição das tensões relacionados as correntes elétricas durante a contração muscular, fornecendo assim em um intervalo de tempo a média desta atividade neuromuscular\cite{reaz2006techniques}.

A sEMG caracteriza-se pela utilização de um dispositivo sobre a pele do paciente, o qual implica a detecção dos potências elétricos relativos as fibras musculares, ou seja é possivel detectar quando um musculo é ativado e qual o movimento foi execultado, e ainda relacionar o associação dos diferentes musculos envolvidos \cite{botelho2010avaliaccao}.

O sinal EMG é registrado normalmente por eletrodos de superfície, mas pode também ser utilizado eletrodos de agulha \cite{eftaxias2015detection}.

\section{Características do sinal}

Primeiramente temos a Unidade motora (UM), uma coleção de fibras musculares inervadas por um único neurônio motor alfa, sendo a menor unidade operacional de um musculo, podendo ser ativada pelo controle neural. A atividade muscular ocorre a ativação do neurônio alfa dentro da UM, produzindo tensão nas fibras musculares ao passo que ocorre a propagação do sinal ao longo das fibras, o musculo é relaxado quano o neurônio alfa para a atividade \cite{yousefi2014characterizing}.

O potencial de ação da unidade motora (PAUM) é a soma dos potenciais de ação das contrações musculares em uma unidade motora, as voltagens detectadas pelos eletrodos da sEMG, descrevem a soma de todas as UM ativas, ou seja a soma de todos os PAUM \cite{yousefi2014characterizing}.

A equação que descreve um sEMG é ~\ref{eq:EMGT}. Um sEMG é a medida ao longo do tempo da contração de um PAUM m, de um total de N PAUMs, a função n(t) representa possiveis ruidos encontrados na coleta, ambas as funções são parametrizadas pelo tempo \cite{yousefi2014characterizing}.

\begin{equation} \label{eq:EMGT}
    EMG_{t} =\sum_{m=1}^{N} PAUMs_{m}(t)+n(t)
\end{equation}

Este tipo de sinal inevitavelmente contera ruído do equipamento, de outras fontes biológicas presentes no corpo do individuo \cite{yousefi2014characterizing}.






Um sEMG comum varia entre 0.1 a 0.5 millivolt, podendo conter frequência de até 10 kHz, a amplitude do sinal depende de vários fatores como o tipo de eletrodo utilizado, nivel do esforço muscular, colocação dos eletrodos.

\section{Usos}
A sEMG é largamente utilizada por pelas diversas áreas cientificas que estudam o movimento humano, como Médicos, Fisioterapeutas, Fonoaudiólogos e profissionais em Educação Física\cite{nascimento2012surface}.

\section{sEMG e machine learning}
A utilização de Aprendisado de máquinas em conjunto com sEMG para o auxilio na detecção de alguma doença, já é um tópico bem explorado possuindo diversas publicações relacionadas a esta área de estudo, na base de dados Capes a string de busca '(emg AND (Machine learning))' retornou 4.217 resultados, no \textit{ScienceDirect} a mesma \textit{string} retornou 2.396 \section{Parkinson e sEMG} e no \textit{ieeexplore} 169.

Como observado na revisão acadêmica de \cite{yousefi2014characterizing}, os transtornos neuromusculares podem são comumente separados em Miopatia e Neuropatia, onde Miopatia refere-se a um grupo de patologias que atigem diretamente o tecido muscular sem relação com disfunções do sistema nevorso. Já a Neuropatia, onde encontra-se a doença de Parkinson, é caracterizada por qualquer dano nos nervos envolvidos no controle muscular.

Em resumo as diversas tecnicas de Aprendizado de máquinas sobre sEMG, pode ser catalogadas em três etapas os dados devem ser analisados, decompostos e classificados\cite{yousefi2014characterizing}.

