\begin{resumo}
    A Doença de Parkinson é uma doença neurológica que atinge uma grande parcela da população mundial, sendo majoritariamente pessoas idosas. Apesar de não terem sido descobertos os motivos efetivos que causam a doença, seu diagnóstico é possível graças aos sinais evidentes causados pela presença da mesma, sendo os mesmos suficientes. Ainda existe uma necessidade de se obter mais informações sobre o estado do paciente dessa doença para possíveis estudos sobre a mesma. Para garantir o acesso à essas informações e auxiliar no diagnóstico da Doença de Parkinson é possível utilizar os conceitos de Aprendizado de Máquina, que tem sido muito requisitado em diversas áreas do conhecimento, incluindo a área da saúde, e para a solução de problemas de diversos contextos. Este trabalho de conclusão de curso implementa um sistema denominado DPDP (Detector Preliminar da Doença de Parkinson) no qual é aplicado um modelo de Aprendizado de Máquina para o auxílio no diagnóstico preliminar da doença de Parkinson. Isso foi feito utilizando amostras recolhidas por dispositivos de coletas de sinais de eletromiográfia de superfície (sEMG), afim de reconhecer padrões e obter novas informações sobre a doença em questão. O sistema utiliza o algoritimo \textit{Random Forest} para detecção dos padrões da doença, pois o mesmo obteve maior exito em comparação a outros modelos estudados neste trabalho.

 \vspace{\onelineskip}
    
 \noindent
 \textbf{Palavras-chaves}: Parkinson; \textit{Machine Learning}; sEMG.
\end{resumo}
